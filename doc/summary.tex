\documentclass{article}
\usepackage[latin1]{inputenc}
\usepackage[ngerman]{babel}
\usepackage{graphicx}
\usepackage{fancyhdr}
\pagestyle{fancy}
\fancyhead[LO]{Seminar: Webprojekt mit PHP und MySQL}
\renewcommand{\headrulewidth}{0.4pt}
\fancyfoot[C]{\thepage}
\fancyfoot[L]{Gennaro Piano}
\fancyfoot[R]{\today}
\renewcommand{\footrulewidth}{0.4pt}
\begin{document}
\section*{Automatische WSUS Auswertung}
Die KMU IT Management AG bietet den Kunden die M�glichkeit, die Windows Update �ber den WSUS Server herunterzuladen und zu installieren. Durch dieses Angebot muss sich der Kunde nicht mehr �ber die Installation der Updates k�mmern. Die WSUS Datenbank ist w�hrend der Lebensdauer gewachsen und so wurden Produktklassen eingef�hrt. Eine Produktklasse ist ein Container der Computerobjekte beinhaltet. Den Produktklassen kann man die Windows Updates zuteilen, dadurch werden auf den Clients nur die Updates installiert, die der dazugeh�rigen Produktklasse zugeteilt worden sind. Ein Client kann selbstverst�ndlich mehreren Produktklassen zugeteilt werden. Es gibt Produktklassen die fast alle Clients beinhalten, als Beispiel w�re zum Beispiel die Produktklasse Office 2010 zu nennen. Diese Konfiguration erm�glicht einen sehr guten Einblick des momentanen Zustands. Die Auswertung der Updates wird hingegen immer komplizierter und dauert jedes Jahr l�nger. Die Auswertung wird jeden Monat manuell erstellt. Die Auswertung ist relativ m�hsam und kostet pro Quartal ein bis zwei Arbeitstage.
F�r jede Produktklasse muss das Update neuaufgenommen werden, danach muss bei jedem Kunde �berpr�ft werden wie viele Updates letztes Quartal installiert worden sind.
\newline \newline
Diese m�hsame Arbeit wird ab n�chsten Monat durch eine Webanwendung massiv verk�rzt. Momentan l�uft die Anwendung in einer Testumgebung. Sie bringt folgende Vorteile:
\begin{itemize}
\item Speichern von Updates und Produktklassen in einer Datenbank �ber das Webinterface
\item Das aufgenommene Update kann �ber das Webinterface an mehrere Produktklassen zugeteilt werden. Dies vermeidet Dateninkonsistenz und unn�tige Redundanz
\item Suchen und Bearbeiten von Updates und Produktklassen �ber das Webinterface
\item Ein Update wird mit allen n�tigen Attributen abgespeichert. Ohne ein Erscheinungsdatum, Knowledgbase Nummer oder Titel wird ein Update nicht in der Datenbank aufgenommen. Dadurch sinkt die Wahrscheinlichkeit auf Dateninkonsistenz.
\end{itemize}
Bevor die Webanwendung produktiv genutzt werden kann, muss die Kundenmaske in der Webanwendung implementiert werden. Die Kundenmaske erm�glicht es neue Kunden aufzunehmen und Ihnen die benutzen Produktklassen zuzuteilen. Nach dieser Erweiterung m�ssen neu erschienene Updates nur einmal aufgenommen werden, den entsprechende Produktklassen zugeteilt werden. Die Auswertung der Daten wird durch SQL Statements vollst�ndig automatisiert sein. Man rechnet mit einer Zeitersparnis von 1.5 Tagen pro Quartal. 

\end{document}