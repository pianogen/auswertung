\documentclass{article}
\usepackage[latin1]{inputenc}
\usepackage[ngerman]{babel}
\usepackage{graphicx}
\usepackage{enumitem}
\usepackage{multirow}
\usepackage{threeparttable}
\usepackage{lscape}

\begin{document}

\title{Automatisierung der WSUS Auswertung
\\ Ein Projekt mit PHP\textbackslash MSSQL}
\author{Gennaro Piano}
\date{\today}

\maketitle

\begin{figure}[htbp]
		\centering
	\includegraphics[width=10cm]{wsus.jpg}
		\label{fig:logo}
\end{figure}

\newpage
\tableofcontents
\newpage
\section{Ausgangslage}
Wir bieten unseren Kunden die M�glichkeit die Windows Updates �ber unserem WSUS Server herunterzuladen. Durch dieses Angebot muss sich der Kunde nicht mehr �ber fehlerhaften Updates k�mmern, da wir die Updates zuerst in einer Testumgebung testen und diese danach freigeben. Um die Updateverwaltung einfach zu halten wurden auf dem WSUS Server Produktklassen erstellt. Computer und Server von Kunden die unser Angebot nutzen m�chten, werden den entsprechenden Produktklassen zugeteilt. Updates die freigegeben sind werden wiederum den entsprechenden Produktklassen zugeteilt und somit auf den Ger�ten der Kunden installiert. Dieses System ist gut durchdacht, jedoch ist es sehr kompliziert diese Daten auszuwerten. \newline \newline
Um die Daten auszuwerten werden jedes Quartal alle freigegebenen Updates manuell in einer Excel Datei aufgenommen. Jedes Update wird danach in die entsprechende Produktklasse kopiert. Auf einem weiterem Excel Blatt wird ausgewertet, welcher Benutzer welche Produktklassen ben�tigt. \newline \newline
Dieser Prozess ist sehr zeitintensiv und kostet pro Quartal circa eins bis zwei Arbeitstage.

\end{document}
